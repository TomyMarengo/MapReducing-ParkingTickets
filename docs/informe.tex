\documentclass[a4paper,12pt]{article}

\usepackage[utf8]{inputenc}
\usepackage[T1]{fontenc}
\usepackage[spanish]{babel}
\usepackage{caption}
\usepackage{amsfonts, amsmath, amssymb, enumitem, authblk, times, framed, varwidth, graphicx, placeins, indentfirst, pdfpages, fancyhdr, titling, listings} 
\usepackage{float}
\usepackage[justification=centering]{caption}
\setlength{\textwidth}{180mm}
\setlength{\textheight}{250mm}
\setlength{\oddsidemargin}{-10mm}
\setlength{\evensidemargin}{15mm}
\setlength{\topmargin}{-10mm}

\renewcommand{\baselinestretch}{1.4}
\renewcommand{\headrulewidth}{0.5pt}

\lhead{\begin{picture}(0,0) \put(0,0){\includegraphics[width=20mm]{./LogoITBA}} \end{picture}}
\renewcommand{\headrulewidth}{0.5pt}

\def\FIG#1#2{%
	{\centering#1\par}
	#2}

\pagestyle{fancy}

\begin{document}
	
	\begin{titlepage}
		\centering
			{\includegraphics[width=0.50\textwidth]{LogoITBA}\par}
			{\bfseries\LARGE Instituto Tecnol\'ogico de Buenos Aires \par}
			\vspace{2cm}
			{\scshape\Huge MapReducing Parking Tickets \par}
			\vspace{0.5cm}
			{\itshape\Large 72.42 Programación Orientada a Objetos - 2024Q1 \par}
			\vspace{1cm}
			
			{\Large \textbf{\underline{Alumnos:}} \par}
			{\Large \large \textbf{Tomás Santiago Marengo}, 61587 \par}
			{\Large \large \textbf{Abril Occhipinti}, 61159 \par}	
			{\Large \large \textbf{Santino Ranucci}, 62092 \par}	
			{\Large \large \textbf{Agustin Zakalik}, 62068 \par}	
			
			\vspace{1cm}
			
			{\Large \textbf{\underline{Profesores:}} \par}
			{\Large \large \textbf{Ing. Marcelo Turrín} \par}
			{\Large \large \textbf{Ing. Franco Román Meola} \par}		
			
			\vfill
	\end{titlepage}
	
	\newpage
	\tableofcontents
	\newpage
	
	\section{Introducción}
	
	\begin{itemize}
		\item Cómo se diseñaron los componentes de cada trabajo MapReduce, qué decisiones se tomaron y con qué objetivos. Además alguna alternativa de diseño que se evaluó y descartó, comentando el porqué.
		\item El análisis de los tiempos para la resolución de cada query: En caso de poder, analizar la diferencia de tiempos de correr cada query aumentando la cantidad de nodos (hasta 5 nodos) en una red local. De no poder, intentar predecir cómo sería el comportamiento.
		\item Potenciales puntos de mejora y/o expansión.
		\item La comparación de los tiempos de las queries ejecutándose con y sin Combiner.
		\item Otro análisis de tiempos de ejecución de las queries utilizando algún otro elemento de optimización a elección por el grupo.
		\item Para todos los puntos anteriores, no olvidar de indicar el tamaño de los archivos utilizados como entrada para las pruebas (cantidad de registros).
	\end{itemize}
	
	
\end{document}
